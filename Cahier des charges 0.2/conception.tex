\documentclass[12pt,a4paper]{article}
\usepackage[french]{babel}
\usepackage[utf8x]{inputenc}
\usepackage{graphicx}
\usepackage[colorinlistoftodos]{todonotes}
\usepackage[T1]{fontenc}

\title{Cahier des Charges}
\author{Groupe C L3SPI}
\date{2016/2017}

\begin{document}
	\begin{titlepage} %titlepage
    \begin{center}
    \includegraphics[scale=0.6]{logo-UBL.jpg}
    \includegraphics[scale=0.19]{logo_ic2.png} 
    \\[5cm]
    \textsc{\Huge \bfseries {Cahier des Charges}}
    \\[3cm]
    \textrm{Par : Groupe C L3SPI} \strut{}\\[.5ex]\strut{}\\[.5ex]
    \textrm{Pierre CHAMPION}\vfill
    \textrm{Weibin WUANG} \vfill
    \textrm{Elliot CANDALE} \vfill
    \textrm{Florian DEZERE} \vfill
    \textrm{Tristan BIARDEAU} \vfill
    \textrm{Marius RIVIERE} \vfill
    \textrm{Ewen CHAUDEMANCHE} \vfill
    \\[3cm]
    
    % Bottom of the page
    \vfill
    \textrm{ Clients : Jacoboni Pierre et Despres Christophe }
    {\newline Annee Universitaire : 2016-2017}
    \end{center}
    \end{titlepage}
    
    \newpage
	\strut
	\newpage
    \tableofcontents
    \newpage
   
    \section{Introduction}
        \subsection{Présentation du document}
        La problématique de notre projet est d'amener un utilisateur lambda à résoudre un sudoku, disons plus ou moins à apprendre un résoudre un sudoku. Pour cela l’utilisateur aura accès à plusieurs modes de jeux; en commençant par une phase de tutoriel qui expliquera le principe du jeu ainsi que les différentes méthodes de résolution des grilles. 


    Les différents objectifs de notre projet sont d’implémenter différents modes de jeux, un mode arcade/aventure ainsi qu’un mode libre; dans le mode arcade l’utilisateur doit commencer au début des niveaux de difficulté, une fois arrivé en jeu, il doit par la suite pouvoir jouer, lors d’une partie il peut être amené à demander de l’aide, la demande de l’aide 
        \vspace{1cm}


        \subsection{Portée}

        Le document est destiné :
        \begin{itemize}
            \item aux enseignants Jacoboni et Depres.
            \item a l'équipe de développement du produit les Dieux incarnés.
        \end{itemize}

        \vspace{0.5cm}
        \newpage
        
    \section{Présentation du projet}
        \subsection{PouetPouetPouet}
		
		\subsection{PouetPouet}
		
        \newpage
        
        
    \section{Objectifs}
        \subsection{SuperPouet!}
		\newpage
	    
	\section{Critères d’acceptabilité du produit}
	    \subsection{PouetPouetPouet}
		\subsection{PouetPouet}
	    \newpage
    
    \section{Les Besoins}
	    \subsection{Les acteurs}
		Le Joueur:
		\begin{itemize}
            \item lancer le tutoriel
            \item lancer une partie
            \item consulter les règles
            \item consulter et modifier les options
            \item quitter le jeu
		\end{itemize}
	
	    \subsection{Scénario d’utilisation}
            \textrm{Le joueur lance le jeu et accède au menu principal ; à partir d’ici, il peut accéder aux différents menus du jeu.}
            \strut{}\\[.5ex]
            \begin{itemize}
                \item Tutoriel :
    Depuis le menu tutoriel, le joueur va pouvoir accéder au règles ainsi que suivre différents scénarios d’apprentissage des méthodes de résolution de sudoku.

                \item Libre :
    Depuis le menu libre, le joueur pourra lancer une partie en fonction de certains paramètres : la difficulté, les méthodes de résolutions,...
                \item Arcade :
    Dans ce menu, le joueur aura accès à des parties à son niveau de jeu, incluant ou non des aides, il pourra augmenter son score pour profiter de niveaux particuliers ou de thèmes de jeu
    
                \item Options :
    Depuis cette fenêtre, le joueur va pouvoir accéder aux différentes options :
    -les couleurs des aides visuelles
    -...

                \item Quitter le jeu :
    Le joueur peut quitter le jeu depuis le menu principal en appuyant sur “Quitter”.
            \end{itemize}
    
    
    
    \newpage
	\section{Maquette}
	    
    	\begin{figure}[htbp]
            \center
            \includegraphics[scale=0.2]{mettreimageici}
            \caption{Ecran titre}
		\end{figure}
		

    
    
\end{document}